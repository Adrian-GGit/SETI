\documentclass[12pt, a4paper]{article}

\begin{document}

\tableofcontents

\section{Einleitung}
Hier eine kurze Einleitung in welchem Rahmen diese Arbeit entstanden ist und schonmal ganz kurz auf SETI eingehen.

\section{SETI Breakthrough Listen}
In dieser Section wird die SETI Challenge detailierter beschrieben. Dabei sollten wir auf folgende Punkte eingehen:
\begin{itemize}
	\item Allgemeine Beschreibung, was ist die Aufgabe bei der SETI Challenge
	\item Beschreibung des Datensatzes (Kadenzen erklären, Umfang und Verteilung des Datensatzes, etc)
\end{itemize}

\section{Implementierung}
Kurze Einleitung in die Implementierung bevor es in die Unterkapitel geht.

\subsection{Tech Stack}
Beschreibung der verwendeten Tools und wichtige Bibliotheken, sofern schon hier sinnvoll

\subsection{Erste Ansätze mit Computer Vision}
Hier können wir unseren ersten Ansätze mit reiner Computer Vision beschreiben und zunächst darauf eingehen, wie sich traditionelle Computer Vision von Machine Learning unterscheidet / abgrenzt und warum wir zunächst mit CV experimentiert haben.

\subsubsection{Implementierte CV Filter}
Hier können wir die Filter, die wir am Anfang implementiert haben jeweils kurz beschreiben

\subsubsection{Zwischenfazit: CV alleine bringt keinen Erfolg}
Kurzes Zwischenfazit, das CV alleine nichts nützt, weil die Signale selbst nach der Anwendung unserer Filter auf vielen Positives nicht zu extrahieren sind.

\section{Deep Learning}
Hier können wir nun gut den Übergang zu Deep Learning und CNNs machen. Auch hier erstmal eine kurze Einführung mit den Grundlagen, bevor es in die Subsections geht. Ich denke hier können wir auch noch den Bezug zur CV mit rein nehmen.

\subsection{Convolutional Neural Networks}
Hier können wir uns nochmal überlegen oder mit Herrn Baier besprechen, wie doll wir bei den Erklärungen zu den einzelnen Punkten von CNNs ins Detail gehen sollen. Vll können wir auch vieles als bekannt voraussetzen und uns mehr auf unsere Implementierung konzentrieren. Dies wäre insgesamt vll nochmal abzuklären auch für die CV Section.

\begin{itemize}
	\item Gewichte
	\item Loss Function
	\item Gradient
	\item Optimizer
	\item Traning und Validierung
	\item Splitten der Trainingsdaten
	\item Dataloaders
	\item Metriken (Accuracy, Precision, Recall, F1 Score, Roc Auc Score)
\end{itemize}

\subsection{Transfer Learning}

\subsubsection{efficientnet}

\subsection{Imbalance}

\subsection{Scheduler}

\subsection{Folds}

\section{Fazit}


\end{document}